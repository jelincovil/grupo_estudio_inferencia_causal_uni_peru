\documentclass[12pt,a4paper,oneside]{book}

% ========================================================================
% PLANTILLA LATEX PARA TESIS - ADAPTADA A GUÍA METODOLÓGICA (MIN CAMBIOS)
% ========================================================================
% Fecha: Enero 2026
% Objetivo: mantener tu estructura, pero agregar lo faltante según guía:
% - "Copia de Documentos" en preliminares
% - "Hipótesis y Variables" en Introducción
% - "Definición de Términos" en Marco Teórico
% - refuerzo mínimo en Metodología
% - "Conclusiones y Recomendaciones"
% - Apéndices: Matriz de Consistencia + Matriz de Operacionalización
% - Bibliografía en estilo APA (biblatex)
% ========================================================================

% ========================================================================
% CONFIGURACIÓN DE ARQUITECTURA Y LENGUAJE
% ========================================================================
\usepackage[utf8]{inputenc}
\usepackage[T1]{fontenc}
\usepackage[spanish,english]{babel}  % último idioma = idioma principal (english)
% Si deseas que el principal sea español, cambia a: \usepackage[english,spanish]{babel}

% Comandos personalizados para cambio rápido de idioma
\newcommand{\texten}[1]{\foreignlanguage{english}{#1}}
\newcommand{\textes}[1]{\foreignlanguage{spanish}{#1}}

% Optimización tipográfica
\usepackage[protrusion=true,expansion=false,final]{microtype}
\microtypecontext{spacing=nonfrench}

% Paquete para programación robusta
\usepackage{etoolbox}

% ========================================================================
% SISTEMA DE METADATOS Y ACCESIBILIDAD DIGITAL
% ========================================================================
\usepackage{qrcode}

% Variables de metadatos del documento
\newcommand{\doctitle}{Título de la Tesis Doctoral}
\newcommand{\docauthor}{Nombre del Autor}
\newcommand{\dockeywords}{palabra clave 1, palabra clave 2, palabra clave 3}
\newcommand{\docsubject}{Tesis Doctoral en Estadística}
\newcommand{\docuniversity}{Universidad Nacional de Ingeniería}
\newcommand{\docfaculty}{Facultad de Ingeniería Económica, Estadística y Ciencias Sociales}

% Configuración avanzada de hyperref
\usepackage{hyperref}
\hypersetup{
    pdftitle={\doctitle},
    pdfauthor={\docauthor},
    pdfsubject={\docsubject},
    pdfkeywords={\dockeywords},
    pdfcreator={LaTeX with hyperref},
    pdfproducer={pdfLaTeX},
    colorlinks=true,
    linkcolor=blue,
    citecolor=blue,
    urlcolor=blue,
    bookmarksnumbered=true,
    bookmarksopen=true,
    bookmarksopenlevel=2,
    pdfstartview={FitH},
    pdfpagemode=UseOutlines,
    pdfpagelayout=OneColumn,
    pdfdisplaydoctitle=true,
    breaklinks=true,
    unicode=true
}

% Comando robusto para generar códigos QR
\newcommand{\qrlink}[2]{%
    \begin{center}
        \qrcode[height=1.5cm]{#1}\\[0.5em]
        \small\texttt{#2}
    \end{center}
}

% ========================================================================
% GEOMETRÍA Y ESPACIADO
% ========================================================================
\usepackage{geometry}
\geometry{
    left=3.5cm,
    right=2.5cm,
    top=3cm,
    bottom=3cm,
    headheight=15pt,
    footskip=1.5cm
}

\usepackage{setspace}
\onehalfspacing

% ========================================================================
% JERARQUÍA DE ENUNCIADOS LÓGICOS Y MATEMÁTICOS
% ========================================================================
\usepackage{amsmath, amssymb, amsthm}
\usepackage{mathtools}

\theoremstyle{plain}
\newtheorem{theorem}{Teorema}[chapter]
\newtheorem{lemma}[theorem]{Lema}
\newtheorem{corollary}[theorem]{Corolario}
\newtheorem{proposition}[theorem]{Proposición}

\theoremstyle{definition}
\newtheorem{definition}[theorem]{Definición}
\newtheorem{axiom}[theorem]{Axioma}
\newtheorem{problem}[theorem]{Problema}

\theoremstyle{remark}
\newtheorem{remark}[theorem]{Observación}
\newtheorem{note}[theorem]{Nota}
\newtheorem{case}[theorem]{Caso de Estudio}
\newtheorem{example}[theorem]{Ejemplo}

\renewcommand{\qedsymbol}{$\blacksquare$}

% ========================================================================
% GESTIÓN BIBLIOGRÁFICA (APA) - MIN CAMBIO
% ========================================================================
\usepackage[
    backend=biber,
    style=apa,
    sorting=nyt,
    maxbibnames=99,
    minbibnames=1,
    maxcitenames=2,
    mincitenames=1,
    doi=true,
    url=false,
    isbn=false,
    eprint=false
]{biblatex}

% Mapeo de idioma APA para español (útil si tus refs están en español)
\DeclareLanguageMapping{spanish}{spanish-apa}

\addbibresource{referencias.bib}

\usepackage{csquotes}

% ========================================================================
% CONTROL DE ELEMENTOS GRÁFICOS Y TABLAS
% ========================================================================
\usepackage{graphicx}
\usepackage{float}
\graphicspath{{figuras/}{diagramas/}{logos/}{./}{documentos/}}

\usepackage{subcaption}

\usepackage{booktabs}
\usepackage{tabularx}
\usepackage{multirow}
\usepackage{longtable}
\usepackage{array}

\usepackage{caption}
\captionsetup{
    font=small,
    labelfont=bf,
    format=plain,
    justification=justified,
    singlelinecheck=true,
    margin=10pt
}
\captionsetup[sub]{
    font=footnotesize,
    labelfont=bf,
    justification=centering
}

% ========================================================================
% SOPORTE PARA ALGORITMOS Y CÓDIGO
% ========================================================================
\usepackage[ruled,vlined,linesnumbered]{algorithm2e}
\SetAlFnt{\small}
\SetAlCapFnt{\small}
\SetAlCapNameFnt{\small}

\usepackage{listings}
\usepackage{xcolor}

\definecolor{codegreen}{rgb}{0,0.6,0}
\definecolor{codegray}{rgb}{0.5,0.5,0.5}
\definecolor{codepurple}{rgb}{0.58,0,0.82}
\definecolor{backcolour}{rgb}{0.95,0.95,0.92}

\lstdefinestyle{mystyle}{
    backgroundcolor=\color{backcolour},
    commentstyle=\color{codegreen},
    keywordstyle=\color{magenta},
    numberstyle=\tiny\color{codegray},
    stringstyle=\color{codepurple},
    basicstyle=\ttfamily\footnotesize,
    breakatwhitespace=false,
    breaklines=true,
    captionpos=b,
    keepspaces=true,
    numbers=left,
    numbersep=5pt,
    showspaces=false,
    showstringspaces=false,
    showtabs=false,
    tabsize=2
}
\lstset{style=mystyle}

% ========================================================================
% UNIDADES Y NÚMEROS CON SIUNITX
% ========================================================================
\usepackage{siunitx}
\sisetup{
    output-decimal-marker = {.},
    group-separator = {,},
    group-minimum-digits = 4,
    per-mode = symbol
}

% ========================================================================
% TIPOGRAFÍA Y TÍTULOS
% ========================================================================
\usepackage{titlesec}

\titleformat{\chapter}[display]
  {\normalfont\huge\bfseries\centering}
  {\chaptertitlename\ \thechapter}
  {10pt}
  {\Huge}
\titlespacing*{\chapter}{0pt}{-20pt}{40pt}

\titleformat{\section}
  {\normalfont\Large\bfseries}
  {\thesection}
  {1em}
  {}

\titleformat{\subsection}
  {\normalfont\large\bfseries}
  {\thesubsection}
  {1em}
  {}

% ========================================================================
% ESTRUCTURA DE CIERRE Y APÉNDICES
% ========================================================================
\usepackage[toc,page]{appendix}
\renewcommand{\appendixname}{Apéndice}
\renewcommand{\appendixtocname}{Apéndices}
\renewcommand{\appendixpagename}{Apéndices}

% ========================================================================
% OTROS PAQUETES ÚTILES
% ========================================================================
\usepackage{enumitem}
\setlist{nosep}

\usepackage{marginnote}
\usepackage{lipsum}

\usepackage{cleveref}
\crefname{theorem}{Teorema}{Teoremas}
\crefname{lemma}{Lema}{Lemas}
\crefname{definition}{Definición}{Definiciones}
\crefname{figure}{Figura}{Figuras}
\crefname{table}{Tabla}{Tablas}
\crefname{equation}{Ecuación}{Ecuaciones}

% ========================================================================
% COMANDOS PERSONALIZADOS ÚTILES
% ========================================================================
\newcommand{\TODO}[1]{\textcolor{red}{\textbf{TODO: #1}}}
\newcommand{\nota}[1]{\marginnote{\small\textit{#1}}}
\newcommand{\vect}[1]{\boldsymbol{#1}}
\newcommand{\mat}[1]{\mathbf{#1}}

\DeclareMathOperator{\Var}{Var}
\DeclareMathOperator{\Cov}{Cov}
\DeclareMathOperator{\Corr}{Corr}
\DeclareMathOperator{\E}{E}
\DeclareMathOperator*{\argmax}{arg\,max}
\DeclareMathOperator*{\argmin}{arg\,min}

% ========================================================================
% INICIO DEL DOCUMENTO
% ========================================================================
\begin{document}

% ------------------------------------------------------------------------
% PORTADA
% ------------------------------------------------------------------------
\begin{titlepage}
    \centering
    \includegraphics[width=0.2\textwidth]{logo_uni.png}\par\vspace{1cm}

    {\scshape\LARGE \docuniversity \par}
    \vspace{0.5cm}
    {\scshape\Large \docfaculty \par}
    \vspace{0.5cm}
    {\scshape\large Unidad de Posgrado \par}

    \vspace{2cm}

    {\bfseries\Large TESIS DOCTORAL \par}

    \vspace{1.5cm}

    {\bfseries\LARGE \doctitle \par}

    \vspace{2cm}

    {\large PARA OPTAR EL GRADO ACADÉMICO DE: \par}
    {\bfseries\large DOCTOR EN CIENCIAS CON MENCIÓN EN ESTADÍSTICA \par}

    \vspace{1.5cm}

    {\large PRESENTADO POR: \par}
    {\bfseries\large \docauthor \par}

    \vfill

    {\large ASESOR: \par}
    {\bfseries\large Dr. Nombre del Asesor \par}

    \vfill

    {\large LIMA - PERÚ \par}
    {\large \the\year \par}
\end{titlepage}

% ------------------------------------------------------------------------
% PÁGINAS PRELIMINARES
% ------------------------------------------------------------------------
\frontmatter
\pagenumbering{roman}

\chapter*{Dedicatoria}
\addcontentsline{toc}{chapter}{Dedicatoria}
\begin{flushright}
    \itshape
    A quienes inspiraron este trabajo...
\end{flushright}

\chapter*{Agradecimientos}
\addcontentsline{toc}{chapter}{Agradecimientos}
\lipsum[1-2]

\chapter*{Resumen}
\addcontentsline{toc}{chapter}{Resumen}
\noindent\textbf{Palabras clave:} \dockeywords

\vspace{1cm}
\lipsum[3]

\chapter*{Abstract}
\addcontentsline{toc}{chapter}{Abstract}
\noindent\textbf{Keywords:} \texten{\dockeywords}

\vspace{1cm}
\texten{\lipsum[3]}

% ------------------------------------------------------------------------
% NUEVO (GUÍA): COPIA DE DOCUMENTOS
% ------------------------------------------------------------------------
\chapter*{Copia de Documentos}
\addcontentsline{toc}{chapter}{Copia de Documentos}

\noindent En esta sección se incorporan los documentos exigidos por la Unidad de Posgrado (cartas, actas,
formatos, constancias u otros). Sustituya este texto por los documentos correspondientes.

% Ejemplo (si insertas un PDF como imagen, requiere que el PDF esté en /documentos):
% \begin{figure}[H]
%   \centering
%   \includegraphics[page=1,width=0.95\textwidth]{documentos/acta_aprobacion.pdf}
% \end{figure}

\tableofcontents
\listoffigures
\listoftables
% \listofalgorithms

% ------------------------------------------------------------------------
% CUERPO PRINCIPAL
% ------------------------------------------------------------------------
\mainmatter
\pagenumbering{arabic}

\chapter{Introducción}

\section{Motivación y Contexto}
\lipsum[4]

\section{Planteamiento del Problema}
\lipsum[5]

\section{Objetivos}

\subsection{Objetivo General}
Desarrollar un marco teórico y metodológico para...

\subsection{Objetivos Específicos}
\begin{enumerate}
    \item Objetivo específico 1
    \item Objetivo específico 2
    \item Objetivo específico 3
\end{enumerate}

% ------------------------------------------------------------------------
% NUEVO (GUÍA): HIPÓTESIS Y VARIABLES
% ------------------------------------------------------------------------
\section{Hipótesis y Variables}

\subsection{Hipótesis}
\begin{itemize}
  \item \textbf{Hipótesis general:} ...
  \item \textbf{Hipótesis específicas:} ...
\end{itemize}

\subsection{Variables y Definiciones Operacionales}
\begin{itemize}
  \item \textbf{Variable dependiente / respuesta:} ...
  \item \textbf{Variables independientes / explicativas:} ...
  \item \textbf{Variables de control (si aplica):} ...
\end{itemize}

\section{Justificación}
\lipsum[6]

\section{Estructura de la Tesis}
Esta tesis está organizada de la siguiente manera...

\chapter{Marco Teórico}

\section{Antecedentes}
\lipsum[7-8]

\section{Fundamentos Matemáticos}

\begin{definition}[Espacio de Probabilidad]\label{def:prob_space}
Un espacio de probabilidad es una terna $(\Omega, \mathcal{F}, P)$ donde:
\begin{itemize}
    \item $\Omega$ es el espacio muestral
    \item $\mathcal{F}$ es una $\sigma$-álgebra sobre $\Omega$
    \item $P: \mathcal{F} \to [0,1]$ es una medida de probabilidad
\end{itemize}
\end{definition}

\begin{theorem}[Teorema Central del Límite]\label{thm:clt}
Sea $(X_n)_{n \geq 1}$ una sucesión de variables aleatorias independientes e idénticamente distribuidas con $\E[X_1] = \mu$ y $\Var(X_1) = \sigma^2 < \infty$. Entonces:
$$
\frac{\sqrt{n}(\bar{X}_n - \mu)}{\sigma} \xrightarrow{d} N(0,1)
$$
donde $\bar{X}_n = \frac{1}{n}\sum_{i=1}^n X_i$.
\end{theorem}

\begin{proof}
La demostración se basa en el método de funciones características...
\end{proof}

Como se establece en la \cref{def:prob_space}, un espacio de probabilidad es fundamental para la teoría.
El \cref{thm:clt} es uno de los resultados más importantes.

% ------------------------------------------------------------------------
% NUEVO (GUÍA): DEFINICIÓN DE TÉRMINOS
% ------------------------------------------------------------------------
\section{Definición de Términos}
\begin{description}[leftmargin=!,labelwidth=5cm]
  \item[Modelo estadístico:] ...
  \item[Estimación:] ...
  \item[Inferencia estadística:] ...
  \item[Validación:] ...
  \item[Error / incertidumbre:] ...
\end{description}

\section{Estado del Arte}
\lipsum[9-10]

\chapter{Metodología}

\section{Diseño de la Investigación}
\lipsum[11]

% ------------------------------------------------------------------------
% REFUERZO MÍNIMO (GUÍA): POBLACIÓN/MUESTRA + PROCEDIMIENTO
% ------------------------------------------------------------------------
\section{Población, Muestra y Fuente de Datos}
\lipsum[12]

\section{Procedimiento y Plan de Análisis}
\lipsum[13]

\section{Técnicas Estadísticas}
\lipsum[14]

\section{Implementación Computacional}

\qrlink{https://github.com/usuario/proyecto}{Código fuente disponible en GitHub}

\begin{lstlisting}[language=Python, caption={Ejemplo de código Python}]
import numpy as np
from scipy import stats

def bootstrap_ci(data, n_bootstrap=1000, alpha=0.05):
    """Calcula intervalo de confianza bootstrap."""
    bootstrap_means = []
    for _ in range(n_bootstrap):
        sample = np.random.choice(data, size=len(data), replace=True)
        bootstrap_means.append(np.mean(sample))

    lower = np.percentile(bootstrap_means, 100 * alpha / 2)
    upper = np.percentile(bootstrap_means, 100 * (1 - alpha / 2))
    return lower, upper
\end{lstlisting}

\chapter{Resultados}

\section{Análisis Descriptivo}
\lipsum[15]

\section{Inferencia Estadística}
\lipsum[16]

\section{Validación del Modelo}
\lipsum[17]

\chapter{Discusión}
\lipsum[18-19]

% ------------------------------------------------------------------------
% AJUSTE (GUÍA): CONCLUSIONES Y RECOMENDACIONES
% ------------------------------------------------------------------------
\chapter{Conclusiones y Recomendaciones}

\section{Conclusiones}
\lipsum[20]

\section{Recomendaciones}
\lipsum[21]

\section{Limitaciones}
\lipsum[22]

\section{Líneas Futuras de Investigación}
\lipsum[23]

% ------------------------------------------------------------------------
% BIBLIOGRAFÍA
% ------------------------------------------------------------------------
\chapter*{Referencias Bibliográficas}
\addcontentsline{toc}{chapter}{Referencias Bibliográficas}
\printbibliography[heading=none]

% ------------------------------------------------------------------------
% APÉNDICES
% ------------------------------------------------------------------------
\begin{appendices}

% ------------------------------------------------------------------------
% NUEVO (GUÍA): MATRICES
% ------------------------------------------------------------------------
\chapter{Matriz de Consistencia}
\noindent Sustituya los puntos suspensivos con el contenido real de su investigación.

\begin{longtable}{p{3.4cm} p{11cm}}
\toprule
\textbf{Elemento} & \textbf{Descripción} \\
\midrule
Problema & ... \\
Objetivo general & ... \\
Objetivos específicos & ... \\
Hipótesis & ... \\
Variables / constructos & ... \\
Metodología (diseño, técnica, procedimiento) & ... \\
\bottomrule
\end{longtable}

\chapter{Matriz de Operacionalización de Variables}
\noindent Sustituya los puntos suspensivos con el detalle real (variables, dimensiones, indicadores e instrumentos).

\begin{longtable}{p{3.2cm} p{3.2cm} p{3.2cm} p{4.6cm}}
\toprule
\textbf{Variable} & \textbf{Dimensión} & \textbf{Indicador} & \textbf{Instrumento / Cálculo} \\
\midrule
... & ... & ... & ... \\
... & ... & ... & ... \\
\bottomrule
\end{longtable}

% ------------------------------------------------------------------------
% TUS APÉNDICES ORIGINALES (SE MANTIENEN)
% ------------------------------------------------------------------------
\chapter{Demostraciones Técnicas}
\lipsum[24]

\chapter{Código Fuente}
\lipsum[25]

\chapter{Datos Suplementarios}
\lipsum[26]

\end{appendices}

\end{document}
